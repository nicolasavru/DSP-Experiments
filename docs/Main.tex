\documentclass[12pt]{article}
\usepackage[left=1in,right=1in, top=1in, bottom=1in]{geometry}
\usepackage[small, bf]{caption}
\usepackage{setspace}
\usepackage{engord}
\usepackage{graphicx}
\usepackage{subfigure}
\usepackage{url}


%--------------------------------TITLE PAGE--------------------------------%

\title{ASSIGNMENT 0: DSP}
\author{Nick and Jordan}
\date{\today}

\begin{document}
\maketitle
%\begin{figure}
%	\begin{center}
%		\includegraphics[width=5in]{./bitmap/fsm_wiring.png}
%	\end{center}
%	\caption{Diagram of a fully evolved, automatically generated finite state machine. Each triangle represents a condition. Red directional arrows indicate the FALSE path while green arrows indicate the TRUE path. Output values are indicated by the character inscribed within each node.}
%\end{figure}
\thispagestyle{empty}
\vspace{1in}
\doublespacing

\pagebreak

%--------------------------------IDENTIFICATION--------------------------------%

Tuesday, August 8th, 2010


Jordan Perr-Sauer\\
Class of 2014\\
Electrical Engineering\\
jordan@jperr.com\\

Nicolas Avrutin\\
Class of 2014\\
Electrical Engineering\\
nicolasavru@gmail.com\\


Stack Overflow\\
http://stackoverflow.com/questions/2063284/what-is-the-easiest-way-to-read-wav-files-using-python-summary

Scott Wilson, Stanford University\\
https://ccrma.stanford.edu/courses/422/projects/WaveFormat/


%--------------------------------DESCRIPTION--------------------------------%

\section{Description}

DSP-Experiments consists of python files that can encode and decode images into sound files such that the image can be seen in the spectrogram of the produced sound. The code also provides a method for viewing the spectrogram of the generated audio.

In our spectrograms, the x-axis represents time (in the audio file) and the y-axis represents frequency. The brightness of a given pixel represents loudness. To encode an image such that it can be viewed in a spectrogram, we first separate the image by it's columns of pixels. Each column of pixels will be represented by a ``tone'' in the output audio stream that lasts for a fixed amount of time. These tones consist of sin waves of varying frequencies, at varying amplitudes, depending on the intensity of each pixel in the column. Each column in the input image is converted into such a tone, and these tones are saved to a wave file, one after another (in order) until the entire image has been processed.


%--------------------------------Compilation------------------------------------------%

\section{Compilation}

You do not need to compile any part of this program. There are some dependencies, however, which you must install prior to execution.

\subsection{Dependencies}

\begin{itemize}
\item Python 2.6 (NumPy and PIL are not yet compatible with Python 3)
\item NumPy (http://numpy.scipy.org/)
\item Python Imaging Library (http://www.pythonware.com/products/pil/)
\item ImageMagick (http://www.imagemagick.org) [convert must be in your PATH]
\end{itemize}
%--------------------------------Execution--------------------------------%

\section{Execution}

To produce sound (as a wav file) from an image:

python imageToWav.py g|c pathToImage pathToWav

The g and c options stand for ``Grayscale'' or ``Color''. Using the c option will result in a 3 channel wav file, where each channel of audio represents one channel of color (red, green, and blue). Using the g option will result in a single channel (mono) wav file and all color data will be lost.

To generate the spectrogram of a produced wav file:

python wavToImage.py pathToWav pathToImage

The format of the image will be derived from the extension you give pathToImage. You can use any format supported by imagemagik.


%--------------------------------Features--------------------------------%

\section{Features}

We support color images by using 3-channel wav files. You can specify grayscale or color using command line arguments at execution. This is discussed in ``Execution.''

%--------------------------------Notes--------------------------------%

\section{Notes}

%--------------------------------Listings-------------------------------%

\section{Listings}

\begin{itemize}
\item imageToWav.py
\item wavToImage.py
\item render.py
\end{itemize}

\end{document}
